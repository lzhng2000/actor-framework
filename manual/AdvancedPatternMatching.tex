\section{Advanced Pattern Matching}
\label{advanced-pattern-matching}

\textbf{Note:} The features described in this Section are not available when using Visual Studio due to limitations of the compiler.
Avoid those features if Windows is one of your target platforms.

The advanced pattern matching features in \lib allow you to match on values and to transform data while matching.
An advanced match case begins with a call to the function \lstinline^on^, which returns an intermediate object providing \lstinline^operator>>^.
The right-hand side of the operator denotes a callback, usually a lambda expression.

The basic syntax shown in \S\,\ref{pattern-matching} deduces all types exclusively from provided callbacks.
When using advanced match cases, the types for the match can be defined explicitly, as shown below.

\begin{lstlisting}
behavior verbose_bhvr1{
  on<int>() >> [](int i) { /*...*/ },
  on<int, float>() >> [](int i, float f) { /*...*/ },
  on<int, int, int>() >> [](int a, int b, int c) { /*...*/ }
};
\end{lstlisting}

The behavior \lstinline^verbose_bhvr1^ is semantically equal to \lstinline^bhvr1^ shown in \S\,\ref{pattern-matching-basics}.
It is worth mentioning that passing the lambdas directly is more efficient, since it allows \lib to select a special-purpose implementation.
The function \lstinline^on^ can be used in two ways.
Either with template parameters only or with function parameters only.
The latter version deduces all types from its arguments and matches for both type and value.
To match for any value of a given type, the template \lstinline^val<T>^ can be used, as shown in the following example.

\begin{lstlisting}
behavior bhvr3{
  on(42) >> [](int i) { assert(i == 42); },
  on("hello world") >> [] { /* ... */ },
  on("print", val<std::string>) >> [](const std::string& what) {
  // ...
  }
};
\end{lstlisting}

\textbf{Note:} The given callback can have less arguments than the pattern.
But it is only allowed to skip arguments from left to right.

\begin{lstlisting}
on<int, float, double>() >> [](double) { /*...*/ }             // ok
on<int, float, double>() >> [](float, double) { /*...*/ }      // ok
on<int, float, double>() >> [](int, float, double) { /*...*/ } // ok
on<int, float, double>() >> [](int i) { /*...*/ } // compiler error
\end{lstlisting}

To avoid redundancy when working with match expressions, \lstinline^arg_match^ can be used as last argument to the function \lstinline^on^.
This causes the compiler to deduce all further types from the signature of any given callback.

\begin{lstlisting}
on<int, int>() >> [](int a, int b) { /*...*/ }
// is equal to:
on(arg_match) >> [](int a, int b) { /*...*/ }
\end{lstlisting}

Note that \lstinline^arg_match^ must be passed as last parameter.
If all types should be deduced from the callback signature, \lstinline^on_arg_match^ can be used, which is a faster alternative for \lstinline^on(arg_match)^.
However, \lstinline^on_arg_match^ is used implicitly whenever a callback is used without preceding match expression.

\subsection{Wildcards}

The type \lstinline^anything^ can be used as wildcard to match any number of any types.
%The \lstinline^constexpr^ value \lstinline^any_vals^ can be used as function argument if \lstinline^on^ is used without template paremeters.
A pattern created by \lstinline^on<anything>()^ or its alias \lstinline^others^ is useful to define a default case.
For patterns defined without template parameters, the \lstinline^constexpr^ value \lstinline^any_vals^ can be used as function argument.
The constant \lstinline^any_vals^ is of type \lstinline^anything^ and is merely syntactic sugar for defining patterns.

\begin{lstlisting}[language=C++]
on<int, anything>() >> [](int i) {
  // tuple with int as first element
},
on(any_vals, arg_match) >> [](int i) {
  // tuple with int as last element
  // "on(any_vals, arg_match)" is equal to "on(anything{}, arg_match)"
},
others >> [] {
  // everything else (default handler)
  // "others" is equal to "on<anything>()" and "on(any_vals)"
}
\end{lstlisting}

\subsection{Projections}

Projections perform type conversions or extract data from a given input.
If a callback expects an integer but the received message contains a string, a projection can be used to perform a type conversion on-the-fly.
This conversion must not have side-effects and must not throw exceptions.
A failed projection is not an error, it simply indicates that a pattern is not matched.
Let us have a look at a simple example.

\begin{lstlisting}
auto intproj = [](const string& str) -> option<int> {
  char* endptr = nullptr;
  int result = static_cast<int>(strtol(str.c_str(), &endptr, 10));
  if (endptr != nullptr && *endptr == '\0') return result;
  return {};
};
message_handler fun {
  on(intproj) >> [](int i) {
    // case 1: successfully converted a string
  },
  [](const string& str) {
    // case 2: str is not an integer
  }
};
\end{lstlisting}

The lambda \lstinline^intproj^ is a \lstinline^string^ $\Rightarrow$ \lstinline^int^ projection, but note that it does not return an integer.
It returns \lstinline^option<int>^, because the projection is not guaranteed to always succeed.
An empty \lstinline^option^ indicates, that a value does not have a valid mapping to an integer.
A pattern does not match if a projection failed.

\textbf{Note}: Functors used as projection must take exactly one argument and must return a value.
The types for the pattern are deduced from the functor's signature.
If the functor returns an \lstinline^option<T>^, then \lstinline^T^ is deduced.
